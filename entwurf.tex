\section{Entwurf}

\subsection{Vergleich existierender XML- und HTML-Editoren}

Beim Entwerfen von Benutzeroberflächen ist es immer ratsam, sie so zu gestalten, wie sie einem Benutzer am vertrautesten ist. Daher habe ich für den Entwurf einige existierende Editoren für XML und HTML in ihrer Optik, dem Aufbau von Bedienelementen und der Funktionalitäten getestet, um für den Entwurf meiner GUI einen Denkanstoß zu bekommen.
Die verschiedenen Editoren sind teils kommerziell und teils Freeware, was sich in dem Umfang der Funktionalität, dem Design und der Arbeitsweise widerspiegelt.

\subsubsection{Adobe Dreamweaver}

Der erste Editor, der hier angeschaut wird, ist der Dreamweaver von Adobe. Der kommerzielle HTML-Editor besitzt zahlreiche Features wie Codevervollständigung, eine Vorschau im eingebauten Browser und mehr. Der Code wird dabei ganz traditionell im Texteditor bearbeitet,
\subsubsection{oxygenxml}
\subsubsection{easy xml Editor}

\subsection{Erster Entwurf}