\section{Grundlagen}

\subsection{Additive Learning Resources - AdLeR}



\subsection{XML}

Damit das Programm sein eigentliches Ziel neben der möglichst benutzerfreundlichen GUI, das Editieren von XML-Dateien, erfüllen kann, muss es das Einlesen und Bearbeiten des XML-Baums beherrschen. Zunächst gehe ich auf den allgemeinen Aufbau von XML-Dateien ein, was mich dann auf die spezielle Schema für das Projekt führt. Danach zeige ich verschiedene Parser auf, ihre Differenzen und welcher am besten für dieses Projekt geeignet ist.
\subsubsection{Aufbau}
XML ist ein Format, um Informationen in einer selbst festgelegten Struktur zu speichern. Die Regel, nach der die Struktur festgelegt ist, wird in einer seperaten Datei angelegt. Dafür gibt es zwei Formate, die XML Schema und DTD, wobei ersteres in der Syntax der XML ähnelt. Mithilfe der Schema oder DTD wird nicht nur die Struktur der XML festgelegt, sie dient auch als Mittel zur Gegenprüfung einer XML-Datei. Stimmt diese mit der festgelegten Struktur überein, ist sie valide.


\subsubsection{Parsen}

\subsection{Implementierung}
\subsubsection{Java-Bibliotheken}
\subsubsection{Entwurf der GUI in IntelliJ}

\subsubsection{Evaluation}
